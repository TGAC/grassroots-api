% generated by Docutils <http://docutils.sourceforge.net/>
\documentclass[a4paper,10pt,english]{article}
\usepackage{fixltx2e} % LaTeX patches, \textsubscript
\usepackage{cmap} % fix search and cut-and-paste in PDF
\usepackage[OT4]{fontenc}
\usepackage[utf8]{inputenc}
\usepackage{ifthen}
\usepackage{babel}
\usepackage{longtable}
\usepackage{array}
\setlength{\extrarowheight}{2pt}
\newlength{\DUtablewidth} % internal use in tables
\usepackage{tabularx}

%%% Custom LaTeX preamble
% PDF Standard Fonts
\usepackage{mathptmx} % Times
\usepackage[scaled=.90]{helvet}
\usepackage{courier}

%%% User specified packages and stylesheets

%%% Fallback definitions for Docutils-specific commands

% providelength (provide a length variable and set default, if it is new)
\providecommand*{\DUprovidelength}[2]{
  \ifthenelse{\isundefined{#1}}{\newlength{#1}\setlength{#1}{#2}}{}
}

% abstract title
\providecommand*{\DUtitleabstract}[1]{\centerline{\textbf{#1}}}

% admonition (specially marked topic)
\providecommand{\DUadmonition}[2][class-arg]{%
  % try \DUadmonition#1{#2}:
  \ifcsname DUadmonition#1\endcsname%
    \csname DUadmonition#1\endcsname{#2}%
  \else
    \begin{center}
      \fbox{\parbox{0.9\textwidth}{#2}}
    \end{center}
  \fi
}

% docinfo (width of docinfo table)
\DUprovidelength{\DUdocinfowidth}{0.9\textwidth}

% title for topics, admonitions and sidebar
\providecommand*{\DUtitle}[2][class-arg]{%
  % call \DUtitle#1{#2} if it exists:
  \ifcsname DUtitle#1\endcsname%
    \csname DUtitle#1\endcsname{#2}%
  \else
    \smallskip\noindent\textbf{#2}\smallskip%
  \fi
}

% topic (quote with heading)
\providecommand{\DUtopic}[2][class-arg]{%
  \ifcsname DUtopic#1\endcsname%
    \csname DUtopic#1\endcsname{#2}%
  \else
    \begin{quote}#2\end{quote}
  \fi
}

% hyperlinks:
\ifthenelse{\isundefined{\hypersetup}}{
  \usepackage[unicode,colorlinks=true,linkcolor=blue,urlcolor=blue]{hyperref}
  \urlstyle{same} % normal text font (alternatives: tt, rm, sf)
}{}
\hypersetup{
  pdftitle={FedStage DRMAA for LSF},
  pdfauthor={Łukasz Cieśnik <lukasz.ciesnik@fedstage.com>, Mariusz Mamoński <mamonski@man.poznan.pl>}
}

%%% Body
\begin{document}

% Document title
\title{FedStage DRMAA for LSF%
  \phantomsection%
  \label{fedstage-drmaa-for-lsf}}
\author{}
\date{}
\maketitle

% Docinfo
\begin{center}
\begin{tabularx}{\DUdocinfowidth}{lX}
\textbf{Author}: &
	Łukasz Cieśnik <\href{mailto:lukasz.ciesnik@fedstage.com}{lukasz.ciesnik@fedstage.com}>, Mariusz Mamoński <\href{mailto:mamonski@man.poznan.pl}{mamonski@man.poznan.pl}> \\
\textbf{Organization}: &
	FedStage System, Poznan Supercomputing and Networking Center \\
\textbf{Contact}: &
	Mariusz Mamonski <\href{mailto:mamonski@man.poznan.pl}{mamonski@man.poznan.pl}> \\
\textbf{Date}: &
	2009-05-05 \\
\textbf{Version}: &
	1.0.5 \\
\textbf{Revision}: &
	2340 \\
\textbf{Copyright}: &
	Copyright (C) 2007-2008 FedStage Systems, Copyright (C) 2009-2010 Poznan Supercomputing and Networking Center \\
\end{tabularx}
\end{center}

\DUtopic[abstract]{
\DUtitle[abstract]{Abstract}

This document describes installation, configuration and usage
of FedStage DRMAA for LSF version 1.0.4.
}

\phantomsection\label{contents}
\pdfbookmark[1]{Contents}{contents}
\tableofcontents



%___________________________________________________________________________

\section*{Introduction%
  \phantomsection%
  \addcontentsline{toc}{section}{Introduction}%
  \label{introduction}%
}

FedStage DRMAA for LSF is an implementation of \href{http://www.gridforum.org/}{Open Grid Forum} \href{http://www.drmaa.org/}{DRMAA}
1.0 (Distributed Resource Management Application API) \href{http://www.ogf.org/documents/GFD.133.pdf}{specification}
for submission and control of jobs to \href{http://www.platform.com/Products/platform-lsf-family}{Platform LSF}.  Using DRMAA,
grid applications builders, portal developers and ISVs can use the same
high-level API to link their software with different cluster/resource
management systems.

This software also enables the integration of \href{http://www.fedstage.com/wiki/FedStage_Computing}{FedStage Computing} with the
underlying LSF system for remote multi-user job submission and control
over Web Services.


%___________________________________________________________________________

\section*{Installation%
  \phantomsection%
  \addcontentsline{toc}{section}{Installation}%
  \label{installation}%
}

To compile and install the library just go to main source directory
and type:
%
\begin{quote}{\ttfamily \raggedright \noindent
\$~./configure~{[}options{]}~\&\&~make\\
\$~sudo~make~install
}
\end{quote}

The library requires LSF version 7.0 or further.
To work with older versions it may require some patching.

Notable \texttt{./configure} script options:
%
\begin{quote}
%
\begin{description}
\item[{\texttt{-{}-with-lsf-inc} LSF\_INCLUDE\_PATH}] \leavevmode 
Path to LSF header files (include dir).  This and \texttt{-{}-with-lsf-lib}
options are unnecessary if \texttt{LSF\_ENVDIR} environment variable is
set correctly (e.g. by \texttt{\$LSF\_TOP/conf/profile.lsf}).

\item[{\texttt{-{}-with-lsf-lib} LSF\_LIBRARY\_PATH}] \leavevmode 
Path to LSF libraries (lib dir).

\item[{\texttt{-{}-with-lsf-static}}] \leavevmode 
Link DRMAA against static LSF libraries instead of shared ones.

\DUadmonition[note]{
\DUtitle[note]{Note}

In LSF 7.0.3 the shared libraries are broken in the way they
have some undefined symbols which should be defined.  Using this
option fixes this problem as static libraries are built correct.
}

\item[{\texttt{-{}-prefix} INSTALLATION\_DIRECTORY}] \leavevmode 
Root directory where FedStage DRMAA for LSF shall be installed.
When not given library is installed alongside with LSF.

\item[{\texttt{-{}-enable-debug}}] \leavevmode 
Compiles library with debugging enabled (with debugging symbols not
stripped, without optimizations, and with many log messages enabled).
Useful when you are to debug DRMAA enabled application
or investigate problems with DRMAA library itself.

\end{description}

\end{quote}

There are no unusual requirements for basic usage of library: ANSI C
compiler and standard make program should suffice.  If you have taken
sources directly from SVN repository or wish to run test-suite you would
need additional \hyperref[developer-tools]{developer tools}.  For further information regarding
GNU build system see the INSTALL file.


%___________________________________________________________________________

\section*{Configuration%
  \phantomsection%
  \addcontentsline{toc}{section}{Configuration}%
  \label{configuration}%
}

During DRMAA session initialization (\texttt{drmaa\_init}) library tries to
read its configuration parameters from locations: \texttt{/etc/lsf\_drmaa.conf},
\texttt{\textasciitilde{}/.lsf\_drmaa.conf} and from file given in \texttt{LSF\_DRMAA\_CONF} environment
variable (if set to non-empty string).  If multiple configuration
sources are present then all configurations are merged with values
from user-defined files taking precedence (in following order:
\texttt{\$LSF\_DRMAA\_CONF}, \texttt{\textasciitilde{}/.lsf\_drmaa.conf}, \texttt{/etc/lsf\_drmaa.conf}).

Currently recognized configuration parameters are:
%
\begin{quote}
%
\begin{description}
\item[{pool\_delay}] \leavevmode 
Amount of time (in seconds) between successive checks of queue(s).

Type: integer, default: 5

\item[{cache\_job\_state}] \leavevmode 
According to DRMAA specification every \texttt{drmaa\_job\_ps()} call should
query DRM system for job state.  With this option one may optimize
communication with DRM.  If set to positive integer \texttt{drmaa\_job\_ps()}
returns remembered job state without communicating with DRM for
\texttt{cache\_job\_state} seconds since last update.  By default library
conforms to specification (no caching will be performed).

Type: integer, default: 0

\item[{wait\_thread}] \leavevmode 
If set to 0 every call to \texttt{drmaa\_wait()} or \texttt{drmaa\_synchronize()} pools
DRM for selected/all jobs. By default library creates additional
thread which checks state of all job for duration of DRMAA session.
\texttt{drmaa\_wait()}/\texttt{drmaa\_synchronize()} calls block until finished job
is found.

Type: integer, default: 1

\item[{job\_categories}] \leavevmode 
Dictionary of job categories.  It's keys are job categories names
mapped to \hyperref[native-specification]{native specification} strings.  Attributes set by job
category can be overridden by corresponding DRMAA attributes or
native specification.  Special category name \texttt{default} is used when
\texttt{drmaa\_job\_category} job attribute was not set.

Type: dictionary with string values, default: empty dictionary

\item[{lsb\_events\_file}] \leavevmode 
The location of the \texttt{lsb.events} file. If set the library polls
the LSF events logfile instead of the LSF deamons.

Type: path, default: none

\end{description}

\end{quote}


%___________________________________________________________________________

\subsection*{Configuration file syntax%
  \phantomsection%
  \addcontentsline{toc}{subsection}{Configuration file syntax}%
  \label{configuration-file-syntax}%
}

Configuration file is in form a dictionary.
Dictionary is set of zero or more key-value pairs.
Key is a string while value could be a string, an integer
or another dictionary.
%
\begin{quote}{\ttfamily \raggedright \noindent
configuration:~dictionary~|~dictionary\_body\\
dictionary:~'\{'~dictionary\_body~'\}'\\
dictionary\_body:~(string~':'~value~',')*\\
value:~integer~|~string~|~dictionary\\
string:~unquoted-string~|~single-quoted-string~|~double-quoted-string\\
unquoted-string:~{[}\textasciicircum{}~\textbackslash{}t\textbackslash{}n\textbackslash{}r:,0-9{]}{[}\textasciicircum{}~\textbackslash{}t\textbackslash{}n\textbackslash{}r:,{]}*\\
single-quoted-string:~'{[}\textasciicircum{}'{]}*'\\
double-quoted-string:~"{[}\textasciicircum{}"{]}*"\\
integer:~{[}0-9{]}+
}
\end{quote}


%___________________________________________________________________________

\section*{Native specification%
  \phantomsection%
  \addcontentsline{toc}{section}{Native specification}%
  \label{native-specification}%
}

DRMAA interface allows to pass DRM dependent job submission options.
Those options may be specified by settings \texttt{drmaa\_native\_specification}
or \texttt{drmaa\_job\_category} job attribute.  \texttt{drmaa\_native\_specification}
accepts space delimited \texttt{bsub} options while \texttt{drmaa\_job\_category} is
name of job category defined in \hyperref[configuration]{configuration} file.  \texttt{-a} and \texttt{bsub}
options which are meant for interactive submission of jobs (\texttt{-I}, \texttt{-Ip},
\texttt{-Is}, \texttt{-K}) are \emph{not} supported.

Attributes set in native specification overrides corresponding DRMAA job
attributes which overrides those set by job category.

\leavevmode
\setlength{\DUtablewidth}{\linewidth}
\begin{longtable}[c]{|p{0.261\DUtablewidth}|p{0.307\DUtablewidth}|}
\caption{Native specification strings with corresponding DRMAA attributes.}\\
\hline
\textbf{%
DRMAA attribute
} & \textbf{%
native specification
} \\
\hline
\endfirsthead
\caption[]{Native specification strings with corresponding DRMAA attributes. (... continued)}\\
\hline
\textbf{%
DRMAA attribute
} & \textbf{%
native specification
} \\
\hline
\endhead
\multicolumn{2}{c}{\hfill ... continued on next page} \\
\endfoot
\endlastfoot

drmaa\_job\_name
 & 
\texttt{-J} job name
 \\
\hline

drmaa\_input\_path
 & 
\texttt{-i} input\_path
 \\
\hline

% 
 & 
\texttt{-is} input\_path
 \\
\hline

drmaa\_output\_path
 & 
\texttt{-o} output path
 \\
\hline

% 
 & 
\texttt{-oo} output\_path
 \\
\hline

drmaa\_error\_path
 & 
\texttt{-e} error path
 \\
\hline

% 
 & 
\texttt{-eo} error\_path
 \\
\hline

drmaa\_start\_time
 & 
\texttt{-b} start time
 \\
\hline

drmaa\_deadline\_time
 & 
\texttt{-t} end\_deadline
 \\
\hline

drmaa\_js\_state
 & 
\texttt{-H}
 \\
\hline

drmaa\_transfer\_files
 & 
\texttt{-f} file\_stage\_op
 \\
\hline

drmaa\_v\_email
 & 
\texttt{-u} mail\_user
 \\
\hline

% 
 & 
\texttt{-B}, \texttt{-N}
 \\
\hline

% 
 & 
\texttt{-m} asked\_hosts
 \\
\hline

% 
 & 
\texttt{-x}
 \\
\hline

% 
 & 
\texttt{-n} min\_proc{[},max\_proc{]}
 \\
\hline

% 
 & 
\texttt{-R} res\_req
 \\
\hline

drmaa\_duration\_hlimit
 & 
\texttt{-c} cpu\_limit
 \\
\hline

drmaa\_wct\_hlimit
 & 
\texttt{-W} runtime\_limit
 \\
\hline

drmaa\_wct\_slimit
 & 
\texttt{-We} estimated\_runtime
 \\
\hline

% 
 & 
\texttt{-M} memory\_limit
 \\
\hline

% 
 & 
\texttt{-D} data\_limit
 \\
\hline

% 
 & 
\texttt{-S} stack\_limit
 \\
\hline

% 
 & 
\texttt{-v} swap\_limit
 \\
\hline

% 
 & 
\texttt{-F} file\_limit
 \\
\hline

% 
 & 
\texttt{-C} core\_limit
 \\
\hline

% 
 & 
\texttt{-p} process\_limit
 \\
\hline

% 
 & 
\texttt{-T} thread\_limit
 \\
\hline

% 
 & 
\texttt{-ul}
 \\
\hline

% 
 & 
\texttt{-U} reservation\_id
 \\
\hline

% 
 & 
\texttt{-ar} reservation\_id
 \\
\hline

% 
 & 
\texttt{-wt} warning\_time
 \\
\hline

% 
 & 
\texttt{-wa} warning\_action
 \\
\hline

% 
 & 
\texttt{-s} signal
 \\
\hline

% 
 & 
\texttt{-q} queue\_name
 \\
\hline

% 
 & 
\texttt{-w} dependency
 \\
\hline

% 
 & 
\texttt{-sp} priority
 \\
\hline

% 
 & 
\texttt{-r}, \texttt{-rn}
 \\
\hline

% 
 & 
\texttt{-G} user\_group
 \\
\hline

% 
 & 
\texttt{-g} job\_group\_name
 \\
\hline

% 
 & 
\texttt{-P} project\_name
 \\
\hline

% 
 & 
\texttt{-Lp} ls\_project\_name
 \\
\hline

% 
 & 
\texttt{-E} pre\_exec\_cmd
 \\
\hline

% 
 & 
\texttt{-Ep} post\_exec\_cmd
 \\
\hline

% 
 & 
\texttt{-app} app\_profile
 \\
\hline

% 
 & 
\texttt{-ext} sched\_options
 \\
\hline

% 
 & 
\texttt{-jsdl} jsdl\_doc
 \\
\hline

% 
 & 
\texttt{-jsdl\_strict} jsdl\_doc
 \\
\hline

% 
 & 
\texttt{-k} checkpoint\_dir
 \\
\hline

% 
 & 
\texttt{-L} login\_shell
 \\
\hline

% 
 & 
\texttt{-sla} service\_class\_name
 \\
\hline

% 
 & 
\texttt{-Z}
 \\
\hline
\end{longtable}


%___________________________________________________________________________

\section*{Release notes%
  \phantomsection%
  \addcontentsline{toc}{section}{Release notes}%
  \label{release-notes}%
}


%___________________________________________________________________________

\subsection*{Changes in 1.0.4 release%
  \phantomsection%
  \addcontentsline{toc}{subsection}{Changes in 1.0.4 release}%
  \label{changes-in-1-0-4-release}%
}
%
\begin{quote}
%
\begin{itemize}

\item Fixed the core limit (-C) parsing in the native specification attribute.

\item Fixed infinite loop on calling drmaa\_wait/drmaa\_synchronize routines after the \texttt{CLEAN\_PERIOD}

\end{itemize}

\end{quote}


%___________________________________________________________________________

\subsection*{Changes in 1.0.3 release%
  \phantomsection%
  \addcontentsline{toc}{subsection}{Changes in 1.0.3 release}%
  \label{changes-in-1-0-3-release}%
}
%
\begin{quote}
%
\begin{itemize}

\item Fixed segfault when \texttt{drmaa\_v\_env} was set.
Now uses \texttt{setenv} and \texttt{unsetenv} calls to modify environ
instead of substituting \texttt{environ} pointer.

\item \texttt{drmaa\_transfer\_files} works (in progress).

\item By default when \texttt{-{}-prefix} is not given at configure time
library is installed alongside with LSF.

\item When waiting for any job or with waiting thread enabled
status of all jobs is pooled from DRM in one LSF API call.

\item New configuration option: \texttt{cache\_job\_state}.

\item More detailed error messages.

\item It now compiles against LSF version 6.0 or futher although it was
not tested at runtime.

\end{itemize}

\end{quote}


%___________________________________________________________________________

\subsection*{Changes in 1.0.2 release%
  \phantomsection%
  \addcontentsline{toc}{subsection}{Changes in 1.0.2 release}%
  \label{changes-in-1-0-2-release}%
}
%
\begin{quote}
%
\begin{itemize}

\item \texttt{drmaa\_remote\_command} and \texttt{drmaa\_v\_argv} are quoted
and not interpreted by shell (e.g. spaces are allowed in command and
arguments).  Jobs are created with \texttt{exec} command i.e. unnecessary
shell process dangling for duration of job was eliminated.

\item \texttt{drmaa\_wifexited} follow refinement on DRMAA Working Group mailing
list - returns 1 only for exit statuses not greater than 128.
Previously it returned 1 for all jobs which were run (not aborted).

\item It has been reported that in some situations job which was recently
submitted is not always immediately visible through LSF API.  There is
now workaround for such behaviour.

\item \texttt{drmaa\_transfer\_files} is ignored because of segfaults produced by it.

\item Bugfixes: Segfault when \texttt{drmaa\_v\_argv} is not set.
Native specification parsing bugs.  Various other segfaults and
memory leaks.

\end{itemize}

\end{quote}


%___________________________________________________________________________

\subsection*{Changes in 1.0.1 release%
  \phantomsection%
  \addcontentsline{toc}{subsection}{Changes in 1.0.1 release}%
  \label{changes-in-1-0-1-release}%
}
%
\begin{quote}

\DUadmonition[note]{
\DUtitle[note]{Note}

Version 1.0.1 of library was previously released with 2.0 version
number.  Afterwards we decided this is misleading and does not follow
versioning scheme established by DRMAA Working Group (i.e. it does not
reflect the version of DRMAA specification implemented by the library).
}
%
\begin{itemize}

\item Many attributes implemented:
%
\begin{itemize}

\item \texttt{drmaa\_start\_time},

\item \texttt{drmaa\_native\_specification},

\item \texttt{drmaa\_transfer\_files},

\item job limits.

\end{itemize}

\item Integrates with \href{http://www.fedstage.com/wiki/AdvanceReservation}{FedStage Advance Reservation Library for LSF}.

\item Job category now points to native specification string
in configuration file instead of job group.

\item Thread safe design.

\item Configuration file(s).

\item Lots of bug fixes.

\item More robust code.

\item Meaningful logging, error messages and codes.

\end{itemize}

\end{quote}


%___________________________________________________________________________

\subsection*{Known bugs and limitations%
  \phantomsection%
  \addcontentsline{toc}{subsection}{Known bugs and limitations}%
  \label{known-bugs-and-limitations}%
}

Library covers all \href{http://www.ogf.org/documents/GFD.133.pdf}{DRMAA 1.0 specification} with exceptions listed
below.  It was successfully tested with \href{http://www.platform.com/Products/platform-lsf-family}{Platform LSF} 7.0.3 on Linux
OS and passes the \href{http://drmaa.org/testsuite.php}{official DRMAA test-suite}.  All mandatory and
nearly all optional job attributes (except job run duration soft limit)
are implemented.

Known limitations:
%
\begin{quote}
%
\begin{itemize}

\item \texttt{\$drmaa\_incr\_ph\$} is replaced only within input, output and error file
paths while according to specification it should be also substituted
in job working directory.

\item Host name is ignored in input, output and error path.
They are always copied from and to submission host.

\item Input file is copied from submission host when it is not present
on execution host even when \texttt{i} was not in transfer files attribute.

\item \texttt{drmaa\_wcoredump()} always returns \texttt{false}.

\end{itemize}

\end{quote}


%___________________________________________________________________________

\section*{Developers%
  \phantomsection%
  \addcontentsline{toc}{section}{Developers}%
  \label{developers}%
}

Core functionality of DRMAA is put into \texttt{drmaa\_utils} library.
This library was created in order to keep consistent common functionality
of \href{http://www.fedstage.com/wiki/FedStage_DRMAA_for_PBS_Pro}{FedStage DRMAA for PBS Pro} and \href{http://www.fedstage.com/wiki/FedStage_DRMAA_for_LSF}{FedStage DRMAA for LSF} library.
As it is independent from any particular DRM you may found this library
useful for developing other DRMAAs.  For detailed information please
take a look at \href{../drmaa_utils/apidoc/html/index.html}{source code documentation}.


%___________________________________________________________________________

\subsection*{Developer tools%
  \phantomsection%
  \addcontentsline{toc}{subsection}{Developer tools}%
  \label{developer-tools}%
}

Although not needed for library user the following tools may be required
if you intend to develop FedStage DRMAA for LSF:
%
\begin{quote}
%
\begin{itemize}

\item GNU autotools (autoconf, automake, libtool),

\item \href{http://www.gnu.org/software/bison/}{Bison} parser generator,

\item \href{http://www.gnu.org/software/gperf/}{gperf} perfect hash function generator,

\item \href{http://research.cs.queensu.ca/~thurston/ragel/}{Ragel} finite state machine compiler,

\item \href{http://docutils.sourceforge.net/}{Docutils} for processing this \texttt{README},

\item \href{http://www.latex-project.org/}{LaTeX} for creating documentation in PDF format,

\item \href{http://www.stack.nl/~dimitri/doxygen/}{Doxygen} for generating source code documentation.

\end{itemize}

\end{quote}


%___________________________________________________________________________

\section*{Contact%
  \phantomsection%
  \addcontentsline{toc}{section}{Contact}%
  \label{contact}%
}

Please send your comments or questions to the following mailing list:
%
\begin{quote}

\url{https://www.fedstage.com/lists/listinfo/drmaa-lsf-users}
(\href{mailto:drmaa-lsf-users@lists.fedstage.com}{drmaa-lsf-users@lists.fedstage.com})

\end{quote}

Please also visit the project webpage to find news and new releases of
our software:
%
\begin{quote}

\url{http://www.fedstage.com/wiki/FedStage_DRMAA_for_LSF}

\end{quote}


%___________________________________________________________________________

\section*{License%
  \phantomsection%
  \addcontentsline{toc}{section}{License}%
  \label{license}%
}

Copyright (C) 2007-2008  FedStage Systems

Licensed under the Apache License, Version 2.0 (the ``License'');
you may not use this file except in compliance with the License.
You may obtain a copy of the License at
%
\begin{quote}

\url{http://www.apache.org/licenses/LICENSE-2.0}

\end{quote}

Unless required by applicable law or agreed to in writing, software
distributed under the License is distributed on an ``AS IS'' BASIS,
WITHOUT WARRANTIES OR CONDITIONS OF ANY KIND, either express or implied.
See the License for the specific language governing permissions and
limitations under the License.

% vim700: spell spelllang=en

% vim: ft=rst

% vim: ts=2 sw=2 et

\end{document}
